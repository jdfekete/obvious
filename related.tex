The proliferation of information visualization toolkits have raised interest
in creating abstractions to create them and share functionnalities. Thus,
there have been several studies in software engineering to describe design
patterns for toolkits and concrete implementations to propose generic uses and
mechanics for toolkits.

When creating a software, design patterns are usefull for they are a good way
to facilitate its construction and to make its architecture clearer,
more abstract and extendable. That is why, the information
visualization community has proposed several generic design patterns
for toolkits and applications to facilitate developers work.

Design patterns have been introduced early in InfoVis with the InfoVis standard
reference model. It derives from the standard Model-View-Controller model and
separates an application in three parts : data model, visualization and view.
This pattern has been widely used in almost every toolkits because it provides
a simple efficient high level architecture. 

Nevertheless, toolkits following this model have implemented the three parts in
different way. For example, on one hand, the data model in InfoVis toolkit is column oriented,
on the other, in Jung it is tuple oriented. That is why, Heer et Al have proposed
twelve designs to describe common patterns and to facilitate software and toolkit
creation. They describe several designs for the data model (column or tuple oriented),
and solutions for visualization and view. However, it only regroups common and efficient design
patterns for visual analytics but does not answer to the question of how combining existing techniques
to simplify developer's choice during a software conception. Certainly,  if the toolkit
prefuse mostly follows all these patterns, others only partially apply them: those designs
can not be directly used to extend and combine existing works.

Attemps have been made to address this problem. For example, the JUNG toolkit proposes
a way to make a JUNG graph usable with prefuse. It is a one shot solution to the basic problem.
Also, Borner has proposed an infrastructure to wrap, reuse and combine existing infovis components.
In practice, there are a lot of components, but they have loose coupling and the infrastruture mainly
translates data model into another to combine components. If it fits for static visualization with
no interaction, it could difficulty work in visual analyctics since we want to avoid copying data
models.

As a conclusion, there has been important work made to provide generic design patterns
for information visualization. However, existing toolkits do not apply them in
an uniform way creating useless complexity. Some initiatives to link toolkits have
emerged but some of then are one shot attempts to bind two toolkits so are not generic
enough or some are only dedicated to static visualizations and are not applicable in our
field. Thus, there is a strong need to create a unified software infrastructure for visual
analytics application. To take all those requirements into account, we propose a meta-toolkit
introducing a unified software infrastructure for visual analytics.






Over the past several years, we have seen the development of a variety
of information visualization toolkits ~\cite{fekete-uist2003, DBLP:conf/chi/HeerCL05, jung2003, DBLP:conf/infovis/Weaver04}.
The choice of one of this toolkit is an prominent step in a software development and can be confusing
for visual analytics software developers. So, in this article, we propose and present
a way to abstract and unify existing information visualization toolkits.

Historically, this proliferation of toolkits can be explained by the fact
that each created toolkit addresses a specific problem and/or is designed
for a specific domain. Thus, there is dispersion in terms of
capabilities, since each toolkit has unique and interesting visualizations
and techniques. For example, some toolkits offer several kinds of
graph layouts ~\cite{DBLP:conf/infovis/Weaver04} when others support very sophisticated coordinated views
without specific graph capabilities ~\cite{DBLP:conf/chi/HeerCL05}.
 
That is why, visual analytics developers are almost immediately confronted with
a crucial decision: the choice of the information visualization toolkit to use.
Since, this choice imposes the data structure and then this data structure imposes capabilities
and techniques designed for it. Thus, if a developer needs a technique available in other(s)
toolkit(s), he has to implement it from scratch: it is a waste of time and a source of errors.
  
To address this problem, instead of creating another ultimate toolkit, after a worhshop
gathering several major authors of toolkits ~\cite{vismaster2008}, we specified and implemented a meta-toolkit named
Obvious. By meta-toolkit, we mean a set of interfaces abstracting services provided by
information visualization toolkits following the InfoVis reference model ~\cite{DBLP:journals/tvcg/HeerA06a}.
  
To provide evidence of the usefulness of Obvious, we have implemented, in Java, Obvious
binding modules implementing interfaces and abstractions defined in the specification
for several toolkits (Infovis toolkit ~\cite{fekete-uist2003}, Prefuse ~\cite{DBLP:conf/chi/HeerCL05} and Jung ~\cite{jung2003}). They have been used
to build some proof-of-concepts examples combining different toolkits and also to create
pieces of software used by a current research project ~\cite{BENZAKEN:2011:INRIA-00532552:1}. During those developments,
we have seen the information visualization community can also benefit from Obvious, since important design
questions emerged, and their answers will improve Infovis reference model

In addition, obvious allows developers to eliminate the crucial choice of the toolkit and to avoid to rewrite
existing functions. The following use case shows it is now possible to combine toolkits.
For example, they can choose a data model from JUNG toolkit for a graph, then query it with prefuse predicates,
use a layout introduced in Infovis toolkit to display it and still used network algorithms introduced in JUNG.
With obvious, there are no more design restrictions imposed by an initial choice for developer.

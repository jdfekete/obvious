a) generic implementation of the InfoVis ref. model

b) social process involving workshop to build consensus

c) selecting emerging consensual patterns

Obvious is organized according to the Information Visualization Reference Model in three main packages: data, visu and view. Additionally, it provides utility classes in the util package.


\jo{It seems to me that there were two categories of contributors to the consensus. Firstly there were the developers of existing comparatively generic vis toolkits (InfoVis Toolkit, Prefuse, Improvise). While they had different approaches to their architecture (as detailed in the previous section), they are all relatively application-agnostic. This has understandably had the main influence on the design of Obvious since they are sufficiently generic to offer wide applicability to other toolkits.  Then there developers who worked with specific types of data or application (Jung, Cytoscape, LandSerf~\cite{wood_terrain_2008}, Mondrian). These data/applications have particular character that shaped the design of the Obvious interface (e.g. need for robust graph handling; need for geospatial raster handling; statistical graphics). Sometimes these application areas fit well with the initial Obvious interface (e.g. JUNG graphs), but some others were a little more problematic (e.g. large geospatial rasters can be modelled as tables, but not particularly efficiently. This raises the design question as to what extent Obvious should accommodate these application areas and to what extent should they adapt their internal architectures to accommodate a common Obvious framework? I think this would lead nicely into the next section that provides details on the Obvious data model.}



\subsection{Obvious Design patterns}

\abstract{ This article describes ``Obvious'': a meta-toolkit that
  abstracts and encapsulates information visualization toolkits
  implemented in the Java language as a way to unify their use and
  postpone the choice of which concrete toolkit(s) to use later-on in
  the development of visual analytics applications.  It also reports
  on the lessons we have learned when wrapping popular toolkits with
  Obvious, namely Prefuse, the InfoVis Toolkit, partly Improvise, JUNG
  and other data management libraries.  We show several examples on
  how the uses of Obvious, showing how the different toolkits can be
  combined while sharing their data models.  We also show how Weka, a
  popular machine-learning toolkits, has be wrapped with Obvious and
  can be used directly with all the other wrapped toolkits. 

  We expect Obvious to start a co-evolution process: Obvious is meant
  to evolve when more components of information visualization systems
  will become consensual. It is also designed to help information
  visualization systems adhere to the best practices to provide a
  higher level of interoperability and leverage the domain of visual
  analytics.
}

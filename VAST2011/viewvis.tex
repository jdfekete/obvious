Unlike the data model, during the workshop, no consensus emerged concerning the Visualization and View models in Obvious. The main reason is that currently different approaches exist among toolkits : the monolithic and the polylithic approaches. So, for the moment, there is no way to create an abstract layer for visualization as we did for the data model. Further discussions and workshops may address this problem.

However, we propose a solution to address this problem for monolithic toolkits. We propose wrapping monolithic components in a black box fed by an Obvious data structure and a map of parameters allowing configuration of the component. For example, it is possible to indicate the X and Y axis columns for a scatter-plot. If a developer needs a non-existing visualization component, it simply adds the choice of an implementation toolkit to create it, then this new visualization will be compatible with all data models. Our solution does not generate extra costs of development.

\jo{Can we sure about the last sentence above?. The discussion of Prefuse in 6.1 is an example of this, but requires the user of Prefuse to develop new visualization components and wrap them in Obvious. This seems like an 'extra cost of development' to me.}
\pierreluc{Wrap them in Obvious is pretty easy (about ten lines of code). So yes, there is a tiny extra cost of development.}

Concerning the view, we choose to implement a simplified version of the camera pattern introduced in \cite{DesignPatternsIV}. The black box concept is still present : a view simply wraps a view component of a targeted toolkit.

\jo{The assertion made in this section suggests that the mono/poly-lithic approaches that vary between existing toolkits have less of an impact on data models than on view models. We should probably make some reference to this distinction in Section 2 so that it does not look like we simply ran out of time in the workshop to consider standardization of view models.}
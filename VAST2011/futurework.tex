
\subsection{Extending Obvious Supported Features}

Obvious aims at covering all the features of an infovis toolkit for
which a consensual interface can be specified.  Relying on a common
reference model helps tremendously but falls short on some
implementation design choices.  Still, a few specific services have
already been mentioned during workshops that could give rise to a
shared implementation:

\begin{enumerate}[noitemsep]
\item selection management: to implement cross-toolkit brushing and
  linking,
\item Mappings of data value to scalar value: this feature is used by
  all the visualizations to map data dimensions to screen coordinates,
  color gradients and many other visual attributes. Because the
  interface of such features is small and well understood, consensus
  is reachable,
\item visualization scale, tick-mark and tick-label management,
\item graph layout computation.
%\item ancillary panels, such as object editors.
\end{enumerate}

While most of those features would be of high value and we feel
consensus can be reached, we have not, as of yet, proposed unifying
designs.  The main reason is that while the core structure of those
services is consensual, they rely on parts which are not yet
consensual, such as the application architecture and more elaborated
view, visualization and interaction models.

\subsection{Adding Additional Toolkits and Languages}

Regarding porting Obvious to other languages, two competing approaches
are being considered.  One involves making the API as language neutral
as possible, the other considers that widespread adoption is only
possible on a particular platform and language if its idioms are
respected.

With the first approach, a wished extension would consist in porting
Obvious to other languages and platforms, such as C++, JavaScript and
C\#.  Obvious has been designed to avoid using idioms too specific to
Java so we believe it could be done without much difficulties.  There
would be at least two benefits: the availability of a meta-toolkit is
these languages to wrap infovis toolkits, and the availability of a
common application programmer's interface (API) for infovis that would
simplify learning and spreading the best practices.  Multi-language
APIs already exist and are popular in recommendations of the W3C. For
example, the Document Object Model API (DOM, see
\url{http://www.w3.org/DOM/}) used to manipulate HTML or XML documents
has official bindings for Java and JavaScript and non-standard
bindings for several of the major languages
(\url{http://www.w3.org/DOM/Bindings}) with slight variations to cope
with the language idioms.

With the second approach, the Obvious design patterns should leverage
the support platform/language conventions and blend as well as
possible with its context. Interoperatbility, which is still strongly
wished, should focus on communication formats and protocols rather
than code similarity in the various languages.


\begin{comment}
% C++ has generic containers in the STL (which is standard)
This
endeavor raises some new challenges: each language and platforms
supposes some specific idioms that are hardly translatable in concepts
of the other languages. Java has a generic collection type, for
instance, that does not map to a standard equivalent in C++. In
translating the design verbatim from a language to another, we would
insure some level of compatibility, but at the expense of
idiosyncrasies in our library, which would preclude widespread
adoption in our target languages. Conversely, adopting the target
language's idioms would preclude interoperability of the Obvious
platform across languages.
\end{comment}

\begin{comment}
% [jdf] I disagree. JViews is more like Piccolo or a structured
% graphics system. It is data-agnostic.  I am not sure how it would
% fit with Obvious.

A true test of how generalizable and unifying Obvious would be
post-hoc integration of a new toolkit. IBM ILOG JViews~\cite{JViews}
is one such example being considered. Because it is a commercial
product, this would have the advantage of bringing considerable
exposure to the unification platform.
\end{comment}

A real-world test of how generalizable and unifying Obvious would be
post-hoc integration with an industrial toolkit.  For example,
IBM ILOG JViews ~\cite{JViews} is a commercial monolithic toolkit and
framework including data models, monolithic visualizations, views, a
graph model, an extensive library of graph drawing algorithms.
%including a hierarchical layout that has no rival on the open source market
Interfacing JViews with Obvious would provide JViews users
access to novel features brought by the research community, and open a
venue for research results to make their way into commercial products.

\subsection{Community Building}

Perhaps the most novel experience we retain from Obvious is the
community-driven process to reach consensus and realize a reference
implementation.  We intend to formalize this process, either through a
formal consortium of with a less formal community-driven process,
depending on the response of the community.

\begin{comment}
 experiment on various means to carry this
community-driven consensus building, to see how it can help
consolidate acquired experience in the craft of toolkit design.


% thb style
In many respects, this process is akin to a
standardization process, only it lacks the industry incentive and
backing. Like open source projects, we shall only count on voluntary
contributions (aside partial of funding from public research grants),
only, in the present state of toolkits, it is much more tempting to
devise and expand one's own toolkit than contribute and make
compromise to use a shared design which still lacks serious adoption.


As an afterthought, we find such consolidation effort is rarely found
in research domains, and yet should surely help make the field more
visible and readable from an outsider perspective.
\end{comment}

\subsection{Obvious and Other Visualizations}

VA often needs to combine infovis with Scientific
Visualization and/or GeoSpatial Visualization.  These two domains are
more mature than infovis for standardization: the Scientific
Visualization community is converging towards using VTK~\cite{VTK} as
a \textit{de facto} standard whereas the GeoSpatial Visualization
community already has a mature GeoSpatial Consortium producing
software\cite{OpenGeospatial}.

Geometrical data structures are much less sophisticated in infovis
than in the two others visualization fields.  Pre-computing complex
geometries and maintaining them dynamically is a main concern in
Scientific Visualization and GeoSpatial Visualization; not so much in
infovis.  Moreover, even if at the abstract level Scientific
Visualization and GeoSpatial Visualization share this concern, at the
implementation level, their geometrical structures are quite
different, adding another level of complexity to the problem.

Currently, the state of the art in combining these visualizations is
to put them side-by-side with coordinated interactions (brushing and
linking, dynamic queries)~\cite{Coord3D}.  This kind of integration 
can be implemented by maintaining separate data structures, separate
visualizations and views; coordination being done through ad-hoc item
identifiers shared across the visualizations and acting as pivots.
From a user-centered viewpoint, unifying the interactions would also
improve the usability of mixed visualization applications.

Combining two or the three fields at the software infrastructure
level will require more discussions and experiments between the
communities and seems like a long-term goal.


\subsection{Extending Obvious Supported Features}

Obvious aims at covering all the features of an infomation
visualization toolkit for which a consensual interface can be
specified.  Progress in this respect is actually more advanced
conceptually than in the implementations.  A few specific services
have already been described that could give rise to a shared
implementation:

\begin{enumerate}
\item value set to scalar mappings: this feature is used across
  visualization definitions, to map screen coordinates, color
  gradients and many other visual dimensions onto data
  dimensions. Because the API of such a feature is somewhat small and
  well known, consensus appears reachable.
\item selection representation (to implement cross-toolkit brushing)
\item some interaction techniques such as zoom and pan, selection...
\item scales
\item ancillary panels, such as object editors.
\end{enumerate}

While most of those features would be of high value and we feel consensus can be reached, we have not, as of yet, proposed unifying designs. The main reason is that while the core structure of those services is consensual, they rely on parts which are not yet consensual, such as the application bus, and more elaborated view, visualization and interaction models.

\subsection{Adding Additional Toolkits and Languages}

A true test of how generalizable and unifying Obvious would be
post-hoc integration of a new toolkit. Discovery ~\cite{Discovery2} is
one such example being under consideration, even though the toolkit is
proprietary toolkit, thereby limiting the potential impact of such an
integration.

Another wished extension would consist in porting the toolkit to other
languages and platforms, such as C++. This endeavor raises some new
challenges: each language and platforms supposes some specific idioms
that are hardly translatable in concepts of the other languages. Java
has a generic collection type, for instance, that does not map to a
standard equivalent in C++. In translating the design verbatim from a
language to another, we would insure some level of compatibility, but
at the expense of idiosyncrasies in our library, which would preclude
widespread adoption in our target languages. Conversely, adopting the
target language's idioms would preclude interoperability of the
Obvious platform across languages.


\subsection{Obvious and Other Visualizations}


\subsection{Community Building}

Perhaps the most novel experience we retain from the Obvious experiment is the community-driven process to reach consensus and realize a reference implementation. 
In many respects, this process is akin to a standardization process, only it lacks the industry incentive and backing. Like open source projects, we shall only count on voluntary contributions (aside partial of funding from public research grants), only, in the present state of toolkits, it is much more tempting to devise and expand one's own toolkit than contribute and make compromise to use a shared design which still lacks serious adoption.

We intend to experiment on various means to carry this "freeform" consensus building, to see how it can help consolidate acquired experience in the art of toolkit design. As an afterthought, we find such consolidation effort is rarely found in research domains, and yet should surely help make the field more visible and readable from an outsider perspective.


\subsection{Extending Obvious Supported Features}

Obvious aims at covering all the features of an information
visualization toolkit for which a consensual interface can be
specified.  Relying on a common reference model helps tremendously but
falls short on some implementation design choices.  Still, a few
specific services have already been mentioned during workshops that
could give rise to a shared implementation:

\begin{enumerate}
\item selection management: to implement cross-toolkit brushing and
  linking,
\item Mappings of data value to scalar value: this feature is used by
  all the visualizations to map data dimensions to screen coordinates,
  color gradients and many other visual attributes. Because the
  interface of such features is small and well understood, consensus
  is reachable,
\item visualization scale, tick-mark and tick-label management,
\item graph layout computation.
%\item ancillary panels, such as object editors.
\end{enumerate}

While most of those features would be of high value and we feel
consensus can be reached, we have not, as of yet, proposed unifying
designs.  The main reason is that while the core structure of those
services is consensual, they rely on parts which are not yet
consensual, such as the application architecture and more elaborated
view, visualization and interaction models.

\subsection{Adding Additional Toolkits and Languages}

A true test of how generalizable and unifying Obvious would be
post-hoc integration of a new toolkit. IBM ILOG JViews ~\cite{JViews} is
one such example being considered. Because it is a commercial product, this would 
have the advantage of bringing considerable exposure to the platform. 

Another wished extension would consist in porting the toolkit to other
languages and platforms, such as C++. This endeavor raises some new
challenges: each language and platforms supposes some specific idioms
that are hardly translatable in concepts of the other languages. Java
has a generic collection type, for instance, that does not map to a
standard equivalent in C++. In translating the design verbatim from a
language to another, we would insure some level of compatibility, but
at the expense of idiosyncrasies in our library, which would preclude
widespread adoption in our target languages. Conversely, adopting the
target language's idioms would preclude interoperability of the
Obvious platform across languages.

\begin{comment}
% thb style
\subsection{Community Building}

Perhaps the most novel experience we retain from Obvious is the
community-driven process to reach consensus and realize a reference
implementation.  In many respects, this process is akin to a
standardization process, only it lacks the industry incentive and
backing. Like open source projects, we shall only count on voluntary
contributions (aside partial of funding from public research grants),
only, in the present state of toolkits, it is much more tempting to
devise and expand one's own toolkit than contribute and make
compromise to use a shared design which still lacks serious adoption.

We intend to experiment on various means to carry this
community-driven consensus building, to see how it can help
consolidate acquired experience in the craft of toolkit design.

As an afterthought, we find such consolidation effort is rarely found
in research domains, and yet should surely help make the field more
visible and readable from an outsider perspective.
\end{comment}

\subsection{Obvious and Other Visualizations}

Visual Analytics often needs to combine Information Visualization with
Scientific Visualization and/or GeoSpatial Visualization.  These two
domains are more mature than information visualization for
standardization: the Scientific Visualization community is converging
towards using VTK~\cite{VTK} as a \textit{de facto} standard whereas
the GeoSpatial Visualization community already has a mature GeoSpatial
Consortium producing
software\footnote{\url{http://www.opengeospatial.org}}.

Geometrical data structures are much less sophisticated in information
visualization than in the two others visualization fields.
Pre-computing complex geometries and maintaining them dynamically is a
main concern in Scientific Visualization and GeoSpatial Visualization;
not so much in Information Visualization.  Moreover, even if at the
abstract level Scientific Visualization and GeoSpatial Visualization
share this concern, at the implementation level, their geometrical
structures are quite different, adding another level of complexity to
the problem.

Currently, the state of the art in combining these visualizations is
to put them side-by-side and allow coordination~\cite{vrvis}.  This
kind of integration can use separate data structures, separate
visualizations and views; coordination can be done through ad-hoc
common identifiers shared across the visualization and acting as
pivots.  From a user-centered viewpoint, unifying the interactions
would also improve the usability of mixed visualization applications.

Combining two or the three fields at the software infrastructure
level will require more discussions and experiments between the
communities and seems like a long-term goal.
